\section{Auswertung}

\subsection{Bestimmung von $C_{sys}$}
Zur Ermittlung der Wäremkapzität des Systems wurde zweimal Benzoesäure verbrannt, durch den Anstieg der Temperatur kann man $\Delta T$ ermitteln. Mit der Formel $\Delta _c U = C_{sys} \Delta T$
kann dann nach Umformen $C_{sys}$ berechnen. Zuerst muss aber noch die molare Verbrennungsenergie $\Delta_c U_{molar}$ in $\Delta_c U$ umgerechnet werden.
\begin{align}
	n          & = m/M                        \\
	\Delta_c U & = \Delta_c U_{molar} \cdot n
\end{align}


\begin{align*}
	\Delta_c U_1[Benz.] & = \Delta_c U_{molar} \cdot n_1 = -13.53 KJ \\
	\Delta_c U_2[Benz.] & = \Delta_c U_{molar} \cdot n_2 = -15.22 KJ
\end{align*}


Jetzt können die einzelnen Werte in die Formel zur Berechnung der Wäremkapzität eingesetzt werden.

\begin{align}
	C_{sys} = \frac{\Delta_c U + Z}{\Delta T}
\end{align}

Wobei $Z$ für die Zündquellen steht, in unserem Fall für den Zünddraht und den Baumwollfaden.

\begin{align*}
	C_{sys}_1 = \frac{-13530J + 80J} {0.904K} = -14873 J/K \\
	C_{sys}_2 = \frac{-15220J + 80J} {1.17K} = -12236 J/K
\end{align*}

Der gemittelte Wert der Wäremkapzität des Systems ist somit $C_{sys} = -13555 J/K$.
%-----------------------------------------------------------------------------------------------------------------------------

\subsection{Bestimmung der molaren Verbrennungsenergie der unbekannten Substanz}

Da jetzt $C_{sys}$ bekannt ist, kann über die Formel (3) $\Delta _c U$ der unbekannten Substanz ermittelt werden.\\


\begin{align*}
	\Delta_c U_1[M] & = C_{sys} \cdot \Delta T_1 = -14233 J \\
	\Delta_c U_2[M] & = C_{sys} \cdot \Delta T_2 = -15737 J
\end{align*}

Um die molare Verbrennungsenergie zu berechnen muss die 2. Formel nach $\Delta_c U_m$ umgeformt werden
\begin{align*}
	\Delta_c U_m  = \frac{\Delta_c U}{n}
\end{align*}

\begin{align*}
	\Delta_c U_m_1  = \frac{-14604 J}{3.6 \cdot 10^{-3}} = -3898 KJ/mol \\
	\Delta_c U_m_1  = \frac{-16148 J}{3.7 \cdot 10^{-3}} = -4153 KJ/mol
\end{align*}

Der gemittelte Wert ist somit $\Delta_c U_m = -4025 KJ/mol$

\newpage
%-----------------------------------------------------------------------------------------------------------------------------

\subsection{Berechnung von $\Delta _f H$}
\begin{enumerate}
	\item aufstellen der Reaktionsgleichung:
	      \begin{align*}
		      C_7H_6O_2(s) + \frac{15}{2} O_2(g) \longrightarrow 7 CO_2(g) + 3 H_2O(l)
	      \end{align*}
	\item $\Delta _c H$ berechnen:
	      \begin{align*}
		      \Delta _c H & = \Delta _c U + \Delta n(g) R T                                   \\
		      \Delta _c H & = -3226000 J/mol - (0.5 mol \cdot R \cdot 298.15K) = -3227 KJ/mol
	      \end{align*}
	\item Berechnung von $\Delta _f H$
	      \begin{itemize}
		      \item Gegeben:
		            \begin{table}[H]
			            \centering
			            \begin{tabular}{ll}
				            $\Delta _f H (CO_2 (g)) = -393.5 KJ/mol$ & $\Delta _f H (H_2O (l)) = -241.8 KJ/mol$ \\
				            $\Delta _v H (H_2O) = 44 KJ/mol$         &
			            \end{tabular}
		            \end{table}
		      \item $\Delta _f H (H_2O (g))$ berechnen:
		            \begin{align*}
			            \Delta _f H (H_2O (l)) = \Delta _f H (H_2O (g)) - \Delta _v H (H_2O) = -285.8 KJ/mol
		            \end{align*}
	      \end{itemize}
	      \begin{align*}
		      \Delta _f H = 3 \cdot \Delta _f H (H_2O (l)) + 7 \cdot \Delta _f H (CO_2 (g)) - \Delta _c H = -384.9 KJ/mol
	      \end{align*}
\end{enumerate}












